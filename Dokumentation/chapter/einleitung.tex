\section{Einleitung}

Nachfolgend ist unsere Ausarbeitung für unser Projekt im Modul  'II2021 Entwicklung eines autonomen Fahrzeugs' zulesen. Im laufe des Projektes haben wir viele verschiedene Lösungsansätze für die einzelnen Teilprobleme durchlaufen. Einige dieser Ansätze und unsere Finalelösung der vorhandenen Problemstellungen sind auf den folgenden Seiten beschrieben. Die ermittelten Ansätze wie es nicht geht sind hier stark reduziert, auch wenn wir damit wahrscheinlich eine eigene Hausarbeit füllen könnten.

\subsection{Ausgangsposition}
    
Gegeben ist der Gazebo Simulator mit einer Simulation des cITIcars und einer Simulationswelt, welche mehrere Teststrecken behinhaltet. Das gegeben Modell des cITIcars besaß zunächst nur einen an der linken Seite angebrachten Time-of-Flight-Sensor und eine zentral nach vorne gerichtete Kamera die auf dem Dach des cITIcars befestigt war. Weiterhin sind bereits verschiedene ROS-Knoten gegeben, unteranderem um die Geschwindigkeit und den Lenkwinkel des Autos zusteuern. Für die Umsetzung ist die Programmiersprache Python in kombination mit dem Framework ROS gegeben. Der Wagen darf beliebig um Sensoren erweitert werden, dazu zählen Kameras, Time-of-Flight und Lidar.

\subsection{Zielsetzung}

Ziel des Projektes ist es die gegebene Teststrecke in möglichst geringer Zeit zu überwinden. Beim Bewältigen der Strecke sind verschiedene Aufgaben bzw. Problem zu lösen. Die erste grundsättzliche Aufgabe des Projekts ist es mit dem Auto den Verkehrslinien zufolgen und entsprechend der vorhandenen Linien   zu lenken. Weiterhin ist es Aufgabe eine Parklücke zuerkennen und dort rückwärts einzuparken. Es gilt Hindernisse auf der Fahrbahn festzustellen und diese zu überholen. Für das Überholen soll ein wechsel der Fahrspur jeweils von rechts nach links und zurück erfolgen. Das Fahrzeug soll seine Geschwindkeit selbständig anpassen können und auch problemlos eine Kreuzung überqueren können. 