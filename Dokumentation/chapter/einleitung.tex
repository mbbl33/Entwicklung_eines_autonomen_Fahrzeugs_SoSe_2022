\section{Einleitung}

Nachfolgend ist unsere Ausarbeitung für unser Projekt im Modul  'II2021 Entwicklung eines autonomen Fahrzeugs' zulesen. Im laufe des Projektes haben viele verschiedene Lösungsansätze für die einzelnen Teilprobleme durchlaufen. Einige dieser Ansätze und unsere Finalelösung der vorhandenen Problemstellungen sind auf den folgenden Seiten beschrieben.

\subsection{Ausgangsposition}
    
Gegeben war der Gazebo Simulator mit einer Simulation des cITIcars und einer Simulationswelt, welche mehreren Teststrecken behinhaltet. Das gegeben Modell des cITIcars besaß zunächst nur einen an der linken Seite angebrachten Time-of-Flight-Sensor und eine zentral nach vorne gerichtete Kamera die auf dem Dach des cITIcars befestigt war.

\subsection{Zielsetzung}

Ziel des Projektes ist es die gegebene Teststrecke in möglichst geringer Zeit zu überwinden. Beim bewältigen der Strecke sind verschiedene Aufgaben bzw. Problem zu lösen. Das erste grundsättzliche Aufgabe des Projekts ist es mit dem Auto den Verkehrslinien zufolgen und entsprechend der vorhandenen Linien   zu lenken. Weiterhin ist es Aufgabe eine Parklücke zuerkennen und dort rückwärts einzuparken. Fotlaufend gilt es Hindernisse auf der Fahrbahn festzustellen und diese zu überholen. Das Fahrzeug soll seine Geschwindkeit selbständig anpassen können und auch problemlos eine Kreuzung überqueren können. 