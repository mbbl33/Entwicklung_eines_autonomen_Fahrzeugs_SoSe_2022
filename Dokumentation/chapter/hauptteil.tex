\section{Hauptteil} % Bitte sinnvolle Überschriften für alle Kapitel und Unterkapitel wählen
\subsection{Unterabschnitt 1}
\begin{quote}
	„Beispielzitat 1“ \cite{knuth:1984} \cite{IEEEexample:texfaq}
\end{quote}

\begin{formal}
	Falls Sie doch mal ein Vollzitat verwenden sollten um zum Beispiel eine Definition nach einem bestimmten Werk anzugeben können Sie das so machen.
	\begin{flushright}
		\textit{--- Fachbuch XY \cite{IEEEexample:texfaq}}
	\end{flushright}
\end{formal}

\begin{table}[H]
	\centering
	\caption{Beispieltabelle}
	\begin{tabular}{c | c}
		\hline 
		\large{Überschrift1} & \large{Überschrift2} \\
		\hline \\
		Inhalt1 & Inhalt2\\
		Inhalt3 & Inhalt4\\
	\end{tabular}

\end{table}

\begin{figure}[H]
	\centering	
	\includegraphics[width=.5\textwidth]{img/sample}
	\caption[Beispielabbildung]{Beispielabbildung}
	\captionsource{So könnte man z.b. die Bildquelle angeben}
	\label{fig:Sample}
\end{figure}
\subsection{Unterabschnitt 2}
\Blindtext
\subsection{Unterabschnitt 3}
\Blindtext\Blindtext
\subsection{Unterabschnitt 4}
\Blindtext\Blindtext

\section{Hauptteil 2}

\subsection{Unterabschnitt}
\blindtext
\subsection{Unterabschnitt}
\blindtext
\subsubsection{Unterunterabschnitt}
\blindtext
\subsubsection{Unterunterabschnitt}
\blindtext
\paragraph{Paragraph}
\blindtext
\paragraph{Paragraph}
\blindtext
\subparagraph{Subparagraph}
\blindtext