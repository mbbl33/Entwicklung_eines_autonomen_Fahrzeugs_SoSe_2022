\documentclass{beamer}

%Style
\usetheme{Warsaw}

\setbeamercolor{normal text}{fg=white,bg=black!90}
\setbeamercolor{structure}{fg=white}

\setbeamercolor{alerted text}{fg=red!85!black}

\setbeamercolor{item projected}{use=item,fg=black,bg=item.fg!35}

\setbeamercolor*{palette primary}{use=structure,fg=structure.fg}
\setbeamercolor*{palette secondary}{use=structure,fg=structure.fg!95!black}
\setbeamercolor*{palette tertiary}{use=structure,fg=structure.fg!90!black}
\setbeamercolor*{palette quaternary}{use=structure,fg=structure.fg!95!black,bg=black!80}

\setbeamercolor*{framesubtitle}{fg=white}

\setbeamercolor*{block title}{parent=structure,bg=black!60}
\setbeamercolor*{block body}{fg=black,bg=black!10}
\setbeamercolor*{block title alerted}{parent=alerted text,bg=black!15}
\setbeamercolor*{block title example}{parent=example text,bg=black!15}
%Packages

\usepackage[utf8]{inputenc}
\usepackage[ngerman]{babel}
\usepackage{graphicx}
\usepackage{booktabs}
\usepackage{xcolor}


\title[]{Entwicklung eines autonomen Fahrzeugs}
\author{Sven Thomas \& Maximilian Biebl}
\institute[THM]{Technische Hochschule Mittelhessen}
\date{}

\begin{document}
\begin{frame}
\titlepage
\begin{center}
\begin{figure}[h]
\centering
\includegraphics[width=0.40\textwidth]{"./img/Auto"}
\label{fig:Auto}
\end{figure}
\end{center}
\end{frame}

\begin{frame}
\frametitle{Inhalt}
\tableofcontents
\end{frame}

\section{Linienerkennung und Lenkung}
\subsection{Erste \glqq naive\grqq Idee und Probleme}

\begin{frame}
\frametitle{Erste \glqq naive\grqq  Idee}
Idee: Aus den Vektoren \glqq Druschnittsvektor\grqq \space bilden  und anhand dessen Lenkwinkel bestimmen
\begin{center}

\begin{figure}[h]
\centering
\includegraphics[width=0.80\textwidth]{"./img/vector_gerade"}
\caption{Erster Gedanke}
\label{fig:Lenkung via Vektoren in Gerade}
\end{figure}

\end{center}
\end{frame}

\begin{frame}
\frametitle{Problem}
\begin{center}
\begin{figure}[h]
\centering
\includegraphics[width=0.80\textwidth]{"./img/vector_zu_rechts"}
\caption{Problem wenn zu weit rechts}
\label{fig:Lenkung via Vektoren zu weit rechts}
\end{figure}
\end{center}
\end{frame}

\begin{frame}
\frametitle{Problem}
\begin{center}
\begin{figure}[h]
\centering
\includegraphics[width=0.80\textwidth]{"./img/vector_kurve"}
\caption{Problem in der Kurve}
\label{fig:Lenkung via Vektoren in Kurve}
\end{figure}
\end{center}
\end{frame}



\subsection{Weiterentwicklung der ersten Idee}

\begin{frame}

\frametitle{Wer brauch Y?}

\begin{center}
\begin{figure}[h]
\centering
\includegraphics[width=0.50\textwidth]{"./img/Y-Yeet"}

\label{fig:Weg mit dem Y}
\end{figure}
\end{center}

\begin{itemize}
\item Erkenntinis: wir brauchen nur X-Koordinaten um Mitte der Linien zubestimmen
\item Anhand des \glqq Durschnitts-X\grqq  und der Bildmitte/Automitte wissen wir in welche richtung wir müssen. 
\item Wie weiter die beiden X-Koordinaten getrennt sind um so stärker müssen wir lenken 
\end{itemize}

\end{frame}

\begin{frame}

\frametitle{Probleme mit Mittellinie}

\begin{center}
\begin{figure}[h]
\centering
\includegraphics[width=0.50\textwidth]{"./img/mittellinie"}
\label{fig:Mittellinie}
\end{figure}
\end{center}

\begin{itemize}
\item Houghline zu empfindlich $\Rightarrow$ zuviel \glqq Beifang\grqq 
\item Houghline zu grob $\Rightarrow$ Probleme bei Kurven
\item einfach Aussenlinie nehmen und 1,25x des Durschnitts-X als Soll-Fahrbahn
\end{itemize}

\end{frame}


\subsection{Lenkwinkelbestimmung}
\begin{frame}

\frametitle{"Discopumper-Algorithmus" für Region-of-Interest}

\begin{itemize}
\item roi bekommt festen Startbereich
\item wir nehmen erstmal alles was wir bekommen an X-Koordinaten
\item wenn wir nichts finden müssen wir breiter werden
\item wenn wir immer noch nichts finden nehmen wir das letzte was wir hatten
\end{itemize}

\begin{center}
\begin{figure}[h]
\centering
\includegraphics[width=0.50\textwidth]{"./img/roi_breite"}
\label{fig:roi}
\caption{Unterschiedlich große roi}
\end{figure}
\end{center}


\end{frame}

\begin{frame}
\frametitle{Verbesserung durch Top-Down view}

\begin{center}
\begin{figure}[h]
\centering
\includegraphics[width=0.50\textwidth]{"./img/longneck"}
\label{fig:longneck}
\end{figure}
\end{center}

\begin{itemize}
\item opimierung durch Top-Down view
\item Kamera ist vor Auto und schaut nach unten
\item könnte realistischer werden, durch Bildtransformation zu einem "pseudo" Top-Down 
\end{itemize}
\end{frame}

\begin{frame}
	\frametitle{Top-Down View}
	\begin{center}
		\begin{figure}[h]
			\centering
			\includegraphics[width=\textwidth]{"./img/topdown _view"}
			\label{fig:topdown}
		\end{figure}
	\end{center}

	\begin{itemize}
		\item \colorbox{red}{rechte Linie}
		\item \colorbox{violet}{errechnete Mittellinie}
		\item \colorbox{white}{\color{black}{Bildmitte $\Rightarrow$ Automitte}}
		\item \colorbox{blue}{1.25-Fache der Bildmitte $\Rightarrow$ Soll-Fahrbahn}
	\end{itemize}
	$\Rightarrow$ $\text{Lenkwinkel} = \text{SollFahrbahn}_X - \text{Bildmitte}_X$

\end{frame}

\end{document}